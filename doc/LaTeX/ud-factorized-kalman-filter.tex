\section{The U-D Factorized Kalman Filter}
\label{The U-D Factorized Kalman Filter}

As should be apparent, preservation of the integrity of the Kalman filter projection and
correction computations is essential, especially when it comes to the $\mathbf{P}$ matrix.
While using a quadratic matrix computation algorithm is clearly an appropriate
implementation strategy, there are times when that is not enough. For instance, if the
numerical range of the numbers challenges the precision range of double-precision
floating-point storage, the behavior of the implemented filter can still be compromised.

One strategy of addressing the numerical range restrictions and floating-point roundoff
errors is to maintain a matrix square root of the $\mathbf{P}$ matrix rather than
$\mathbf{P}$ itself. Basically this compands the dynamic range of $\mathbf{P}$ into a
more manageable range (the square root of very large numbers become smaller numbers, and
the square root of very small numbers become larger numbers). It also guarantees that
$\mathbf{P}$ remains symmetric and positive definite.

In practice, instead of using a matrix square root for $\mathbf{P}$, a factorized form of
$\mathbf{P}$ is used. Factorized algorithms provide better numerical stability, and they
avoid the need to compute square root values. One of the more popular approaches is to use
U-D factorization.

We begin the U-D Kalman filter algorithm description by identifying the items to be
maintained:

\begin{tabular}{l l}
$\hat{\mathbf{x}}_k$ & state estimate at time event $k$ \\
$\mathbf{U}_{P,k}$   & U factor of $\mathbf{P}_k = \mathbf{U}_{P,k} \mathbf{D}_{P,k} \mathbf{U}_{P,k}^T$ at time event $k$ \\
$\mathbf{D}_{P,k}$   & D factor of $\mathbf{P}_k = \mathbf{U}_{P,k} \mathbf{D}_{P,k} \mathbf{U}_{P,k}^T$ at time event $k$
\end{tabular}

The procedure relies on two subalgorithms: \textit{UDProject} and \textit{UDCorrect}.
Both of these subalgorithms are specified below. In addition, the U-D Kalman filter
algorithm relies on $\mathbf{R}_k$ being diagonal, so the diagonalization procedure given
in \S\ref{Uncorrelated Observations} should be used if the observations are correlated.

I. Projection (\textit{a priori}) update:

\begin{tabular}{l l}
\phantom{.} & Compute $\hat{\mathbf{x}}_{k|k-1} = \mathbf{\Phi}_{k|k-1} \hat{\mathbf{x}}_{k-1}$ \\
\phantom{.} & Factor $\mathbf{Q}_{k-1}$ into $\mathbf{U}_{Q,k-1}$ and $\mathbf{D}_{Q,k-1}$ such that
    $\mathbf{Q}_{k-1} = \mathbf{U}_{Q,k-1} \mathbf{D}_{Q,k-1} \mathbf{U}_{Q,k-1}^T$\\
\phantom{.} & \phantom{MM} (this can be done just once if $\mathbf{\Gamma}$ and $\mathbf{Q}$ are static) \\
\phantom{.} & Form $\mathbf{F}_k = \mathbf{\Phi}_{k|k-1} \mathbf{U}_{P,k-1}$\\
\phantom{.} & Form $\mathbf{G}_k = \mathbf{\Gamma}_{k|k-1} \mathbf{U}_{Q,k-1}$\\
\phantom{.} & Use \textit{UDProject} to propagate $\mathbf{U}_{P,k-1}$ and $\mathbf{D}_{P,k-1}$
    to $\mathbf{U}_{P,k|k-1}$ and $\mathbf{D}_{P,k|k-1}$
\end{tabular}

II. Correction (\textit{a posteriori}) update:

\begin{tabular}{l l}
\phantom{.} & $\mathbf{x}^{(0)} = \hat{\mathbf{x}}_{k|k-1}$ \\
\phantom{.} & $\mathbf{U}_P^{(0)} = \mathbf{U}_{P,k|k-1}$ \\
\phantom{.} & $\mathbf{D}_P^{(0)} = \mathbf{D}_{P,k|k-1}$ \\
\phantom{.} & \textbf{for} $i = 1$ \textbf{to} $m$ \textbf{do} \\
\phantom{.} & \phantom{MM} Form ${\delta} = {z}^{(i)} - \mathbf{h}^{(i)} \mathbf{x}^{(i-1)}$ \\
\phantom{.} & \phantom{MM} Use \textit{UDCorrect} to update $\mathbf{x}^{(i-1)}$, $\mathbf{U}_P^{(i-1)}$,
    and $\mathbf{D}_P^{(i-1)}$ to $\mathbf{x}^{(i)}$, $\mathbf{U}_P^{(i)}$, and $\mathbf{D}_P^{(i)}$ \\
\phantom{.} & \textbf{end for} \\
\phantom{.} & $\hat{\mathbf{x}}_{k} = \mathbf{x}^{(m)}$ \\
\phantom{.} & $\mathbf{U}_{P,k} = \mathbf{P}_P^{(m)}$ \\
\phantom{.} & $\mathbf{D}_{P,k} = \mathbf{D}_P^{(m)}$
\end{tabular}

The \textit{UDProject} subalgorithm is based on Thornton’s Modified Weighted Gram-Schmidt (MWGS)
Orthogonalization procedure \cite{thornton1976}.

Recall the Kalman covariance projection update:

\begin{equation*}
    \mathbf{P}_{k|k-1} = \mathbf{\Phi}_{k|k-1} \mathbf{P}_{k-1} \mathbf{\Phi}_{k|k-1}^T + \mathbf{\Gamma}_{k|k-1} \mathbf{Q}_{k-1} \mathbf{\Gamma}_{k|k-1}^T
\end{equation*}

Substituting the U-D factors for $\mathbf{P}$ and $\mathbf{Q}$ gives

\begin{equation*}
    \begin{aligned}
        & \mathbf{U}_{P,k|k-1} \mathbf{D}_{P,k|k-1} \mathbf{U}_{P,k|k-1}^T \\
        & \phantom{MMM} = \mathbf{\Phi}_{k|k-1} \mathbf{U}_{P,k-1} \mathbf{D}_{P,k-1} \mathbf{U}_{P,k-1}^T \mathbf{\Phi}_{k|k-1}^T
        + \mathbf{\Gamma}_{k|k-1} \mathbf{U}_{Q,k-1} \mathbf{D}_{Q,k-1} \mathbf{U}_{Q,k-1}^T \mathbf{\Gamma}_{k|k-1}^T
    \end{aligned}
\end{equation*}

Define

\begin{equation*}
    \begin{aligned}
        & \mathbf{F} = \mathbf{\Phi}_{k|k-1} \mathbf{U}_{P,k-1} \\
        &\phantom{.} \\
        & \mathbf{G} = \mathbf{\Gamma}_{k|k-1} \mathbf{U}_{Q,k-1}
    \end{aligned}
\end{equation*}

Then

\begin{equation*}
    \mathbf{U}_{P,k|k-1} \mathbf{D}_{P,k|k-1} \mathbf{U}_{P,k|k-1}^T
    = \mathbf{F} \mathbf{D}_{P,k-1} \mathbf{F}^T
    + \mathbf{G} \mathbf{D}_{Q,k-1} \mathbf{G}^T
\end{equation*}

Define the block matrices

\begin{equation*}
    \begin{aligned}
        &\mathbf{A} =
        \begin{bmatrix}
         \mathbf{F} & \mathbf{G}
        \end{bmatrix} \\
        &\phantom{.} \\
        &\mathbf{W} =
        \begin{bmatrix}
         \mathbf{D}_{P,k-1} & \mathbf{0} \\
         \mathbf{0} & \mathbf{D}_{Q,k-1}
        \end{bmatrix}
    \end{aligned}
\end{equation*}

Then

\begin{equation*}
    \mathbf{U}_{P,k|k-1} \mathbf{D}_{P,k|k-1} \mathbf{U}_{P,k|k-1}^T = \mathbf{A} \mathbf{W} \mathbf{A}^T
\end{equation*}

The Thornton MWGS algorithm performs a weighted Gram-Schmidt orthogonalization on the row
vectors of $\mathbf{A}$ using $\mathbf{W}$ as the weighting matrix. The result is a unit
lower triangular matrix, $\mathbf{L}$ and a diagonal coefficient matrix $\mathbf{D}_L$,
such that

\begin{equation*}
    \begin{aligned}
        \mathbf{U}_{P,k|k-1} &= \mathbf{L}^T \\
        \mathbf{D}_{P,k|k-1} &= \mathbf{D}_L
    \end{aligned}
\end{equation*}

It should be noted that, although the primary purpose of Gram-Schmidt orthogonalization is
to compute an optimal set of orthogonal basis vectors for a general set of basis vectors,
it is the transformation matrices $\mathbf{L}$ and $\mathbf{D}_L$ that get produced as
by-products of the procedure that we’re interested in here.

The \textit{UDProject} subalgorithm of Algorithm \ref{alg:UDProject} below implements the
Thornton MWGS algorithm. The inputs to the subalgorithm are

\begin{tabular}{l l}
\phantom{.} & $\mathbf{F} = \mathbf{\Phi}_{k|k-1} \mathbf{U}_{P,k-1}$ \\
\phantom{.} & $\mathbf{G} = \mathbf{\Gamma}_{k|k-1} \mathbf{U}_{Q,k-1}$ \\
\phantom{.} & $\mathbf{D}_{P,k-1}$ \\
\phantom{.} & $\mathbf{D}_{Q,k-1}$
\end{tabular}

and the outputs are:

\begin{tabular}{l l}
\phantom{.} & $\mathbf{U}_{P,k|k-1}$ \\
\phantom{.} & $\mathbf{D}_{P,k|k-1}$
\end{tabular}

\begin{algorithm}
    \caption{UDProject}
    \label{alg:UDProject}
    \begin{algorithmic}
        \STATE Use $\mathbf{d}$ as an $n \times 1$ scratchpad vector
        \FOR{$i = n$ \textbf{to} $1$ \textbf{by} $-1$}
            \STATE $\sigma = 0$
            \FOR{$j = 1$ \textbf{to} $n$}
                \STATE $\sigma = \sigma + \mathbf{F}_{ij} * \mathbf{D}_{P,jj} * \mathbf{F}_{ij}$
            \ENDFOR
            \FOR{$j = 1$ \textbf{to} $p$}
                \STATE $\sigma = \sigma + \mathbf{G}_{ij} * \mathbf{D}_{Q,jj} * \mathbf{G}_{ij}$
            \ENDFOR
            \STATE $\mathbf{d}_{i} = \sigma$
            \STATE $\mathbf{U}_{P,ii} = 1$
            \FOR{$j = 1$ \textbf{to} $i-1$}
                \STATE $\sigma = 0$
                \FOR{$k = 1$ \textbf{to} $n$}
                    \STATE $\sigma = \sigma + \mathbf{F}_{ik} * \mathbf{D}_{P,kk} * \mathbf{F}_{jk}$
                \ENDFOR
                \FOR{$k = 1$ \textbf{to} $p$}
                    \STATE $\sigma = \sigma + \mathbf{G}_{ik} * \mathbf{D}_{Q,kk} * \mathbf{G}_{jk}$
                \ENDFOR
                \STATE $\mathbf{U}_{P,ji} = \sigma / \mathbf{d}_{i}$
                \FOR{$k = 1$ \textbf{to} $n$}
                    \STATE $\mathbf{F}_{jk} = \mathbf{F}_{jk} - \mathbf{U}_{P,ji} * \mathbf{F}_{ik}$
                \ENDFOR
                \FOR{$k = 1$ \textbf{to} $p$}
                    \STATE $\mathbf{G}_{jk} = \mathbf{G}_{jk} - \mathbf{U}_{P,ji} * \mathbf{G}_{ik}$
                \ENDFOR
            \ENDFOR
        \ENDFOR
        \FOR{$i = 1$ \textbf{to} $n$}
            \STATE $\mathbf{D}_{P,ii} = \mathbf{d}_{i}$
        \ENDFOR
    \end{algorithmic}
\end{algorithm}

The \textit{UDCorrect} subalgorithm is based on Bierman’s U-D factorization method for
observational updates \cite{bierman1977}. Bierman’s method factorizes the covariance update
of the $i^{\text{th}}$ step of the sequential update algorithm:

\begin{equation*}
    \begin{aligned}
        \mathbf{P}^{(i)} &= \mathbf{P}^{(i-1)} - \mathbf{k}^{(i)} \mathbf{h}^{(i)} \mathbf{P}^{(i-1)} \\
        &= \mathbf{P}^{(i-1)} - \frac { \mathbf{P}^{(i-1)} {\mathbf{h}^{(i)}}^T \mathbf{h}^{(i)} \mathbf{P}^{(i-1)}} { \mathbf{h}^{(i)} \mathbf{P}^{(i-1)} {\mathbf{h}^{(i)}}^T + r^{(i)} }  \\
    \end{aligned}
\end{equation*}

Factoring this into U-D components gives

\begin{equation*}
    \mathbf{U}^{(i)} \mathbf{D}^{(i)} {\mathbf{U}^{(i)}}^T 
    = \mathbf{U}^{(i-1)}
    \left[ \mathbf{D}^{(i-1)}
    - \frac { \mathbf{D}^{(i-1)} \mathbf{v}^{(i)} {\mathbf{v}^{(i)}}^T \mathbf{D}^{(i-1)}} { {\mathbf{v}^{(i)}}^T  \mathbf{D}^{(i-1)} \mathbf{v}^{(i)} + r^{(i)} } \right]
    {\mathbf{U}^{(i-1)}}^T
\end{equation*}

where $\mathbf{v}^{(i)}$ is the $n \times 1$ column vector

\begin{equation*}
    \mathbf{v}^{(i)} = {\mathbf{U}^{(i-1)}}^T {\mathbf{h}^{(i)}}^T
\end{equation*}

Bierman’s method factorizes the inner expression into U-D factors

\begin{equation*}
    \mathbf{D}^{(i-1)}
    - \frac { \mathbf{D}^{(i-1)} \mathbf{v}^{(i)} {\mathbf{v}^{(i)}}^T \mathbf{D}^{(i-1)}} { {\mathbf{v}^{(i)}}^T  \mathbf{D}^{(i-1)} \mathbf{v}^{(i)} + r^{(i)} }
    = \mathbf{B}^{(i)} \mathbf{D}^{(i)} {\mathbf{B}^{(i)}}^T
\end{equation*}

such that

\begin{equation*}
    \mathbf{U}^{(i)} \mathbf{D}^{(i)} {\mathbf{U}^{(i)}}^T = \mathbf{U}^{(i-1)} \mathbf{B}^{(i)} \mathbf{D}^{(i)} {\mathbf{B}^{(i)}}^T {\mathbf{U}^{(i-1)}}^T
\end{equation*}

Hence,

\begin{equation*}
    \mathbf{U}^{(i)} = \mathbf{U}^{(i-1)} \mathbf{B}^{(i)}
\end{equation*}

The \textit{UDCorrect} subalgorithm of Algorithm \ref{alg:UDCorrect} below implements the
U-D factorized correction procedure. The inputs to the subalgorithm are

\begin{tabular}{l l}
\phantom{.} & $\mathbf{x}^{(i-1)}$ \\
\phantom{.} & $\mathbf{U}_P^{(i-1)}$ \\
\phantom{.} & $\mathbf{D}_P^{(i-1)}$ \\
\phantom{.} & $\mathbf{h}^{(i)}$ \\
\phantom{.} & ${\delta} = {z}^{(i)} - \mathbf{h}^{(i)} \mathbf{x}^{(i-1)}$
\end{tabular}

and the outputs are:

\begin{tabular}{l l}
\phantom{.} & $\mathbf{x}^{(i)}$ \\
\phantom{.} & $\mathbf{U}_P^{(i)}$ \\
\phantom{.} & $\mathbf{D}_P^{(i)}$
\end{tabular}

\begin{algorithm}
    \caption{UDCorrect}
    \label{alg:UDCorrect}
    \begin{algorithmic}
        \STATE Use $\mathbf{k}$ as an $n \times 1$ scratchpad vector
        \FOR{$j = 1$ \textbf{to} $n$}
            \STATE $\mathbf{k}_j = \mathbf{h}_j$
            \FOR{$i = 1$ \textbf{to} $j-1$}
                \STATE $\mathbf{k}_j = \mathbf{k}_j + \mathbf{U}_{P,ij} * \mathbf{h}_i$
            \ENDFOR
        \ENDFOR
        \STATE $\sigma = r$
        \FOR{$j = 1$ \textbf{to} $n$}
            \STATE $v = \mathbf{k}_j$
            \STATE $\mathbf{k}_j = \mathbf{D}_{P,jj}  * \mathbf{k}_j$
            \STATE $\omega = \mathbf{k}_j$
            \FOR{$i = 1$ \textbf{to} $j-1$}
                \STATE $\tau = \mathbf{U}_{P,ij}  * \omega$
                \STATE $\mathbf{U}_{P,ij} = \mathbf{U}_{P,ij} - \mathbf{k}_i * v / \sigma$
                \STATE $\mathbf{k}_i = \mathbf{k}_i + \tau$
            \ENDFOR
            \STATE $\mathbf{D}_{P,jj} = \mathbf{D}_{P,jj} * \sigma$
            \STATE $\sigma = \sigma + \omega * v$
            \STATE $\mathbf{D}_{P,jj} = \mathbf{D}_{P,jj} / \sigma$
        \ENDFOR
        \STATE $\varepsilon = \delta / \sigma$
        \FOR{$i = 1$ \textbf{to} $n$}
            \STATE $\mathbf{x}_i = \mathbf{x}_i + \mathbf{k}_i * \varepsilon$
        \ENDFOR
    \end{algorithmic}
\end{algorithm}

