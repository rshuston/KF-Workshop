\section{Relationship to Least-Squares Estimation}
\label{Relationship to Least-Squares Estimation}

As stated previously, the Kalman filter is not a specialized least-squares estimator;
it is its own formulation. However, there is an interesting relationship between the
Kalman filter and the least-squares estimator.

Recall that the least-squares estimation problem relies only on the observation relation
and does not involve the progression of time. The system description is static, and so
the unforced system process model is

\begin{equation*}
    \mathbf{x}_{k+1} = \mathbf{I} \, \mathbf{x}_k + \mathbf{0}
\end{equation*}

Since the progression of time is not applicable, we assume that all observations occur
simultaneously at $k = 1$, and so the observation equation is an overdetermined relationship
of the form

\begin{equation*}
    \mathbf{z}_1 = \mathbf{H}_1 \mathbf{x}_1 + \mathbf{v}_1
\end{equation*}

where $E \left\{ \mathbf{v}_1 \mathbf{v}_1^T \right\} = \mathbf{R}_1$. In addition,
we have no \textit{a priori} knowledge of $\mathbf{x}$ and so $\hat{\mathbf{x}}_0 = \mathbf{0}$,
and the associated covariance is $\mathbf{P}_0 = \mathbf{\infty}$.

Therefore, using the information form of the Kalman filter, the state error covariance becomes

\begin{equation*}
    \begin{aligned}
        \mathbf{P}_1^{-1} &= \mathbf{P}_0^{-1} + \mathbf{H}_1^T \mathbf{R}_1^{-1} \mathbf{H}_1 \\
        &= \left( \mathbf{\infty} \right)^{-1} + \mathbf{H}_1^T \mathbf{R}_1^{-1} \mathbf{H}_1 \\
        &= \mathbf{0} + \mathbf{H}_1^T \mathbf{R}_1^{-1} \mathbf{H}_1 \\
        &= \mathbf{H}_1^T \mathbf{R}_1^{-1} \mathbf{H}_1
    \end{aligned}
\end{equation*}

and the Kalman gain becomes

\begin{equation*}
    \begin{aligned}
        \mathbf{K}_1 &= \mathbf{P}_1 \mathbf{H}_1^T \mathbf{R}_1^{-1} \\
        &= \left( \mathbf{H}_1^T \mathbf{R}_1^{-1} \mathbf{H}_1 \right)^{-1} \mathbf{H}_1^T \mathbf{R}_1^{-1}
    \end{aligned}
\end{equation*}

Then, the Kalman \textit{a posteriori} update of the state estimate becomes a least-squares
solution with the optimal weight $\mathbf{W} = \mathbf{R}_1^{-1}$:

\begin{equation*}
    \begin{aligned}
        \hat{\mathbf{x}}_1 &= \hat{\mathbf{x}}_0 + \mathbf{K}_1 \left[ \mathbf{z}_1 - \hat{\mathbf{x}}_0 \right] \\
        &= \mathbf{0} + \mathbf{K}_1 \left[ \mathbf{z}_1 - \mathbf{0} \right] \\
        &= \mathbf{K}_1 \mathbf{z}_1 \\
        &= \left( \mathbf{H}_1^T \mathbf{R}_1^{-1} \mathbf{H}_1 \right)^{-1} \mathbf{H}_1^T \mathbf{R}_1^{-1} \mathbf{z}_1
    \end{aligned}
\end{equation*}

Hence, when applied to a static system description, the Kalman filter reduces to a
least-squares solution. So I’ve got that going for me, which is nice.
