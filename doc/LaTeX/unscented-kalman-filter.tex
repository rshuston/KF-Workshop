\section{The Unscented Kalman Filter}
\label{The Unscented Kalman Filter}

Consider the nonlinear system dynamical process model

\begin{equation*}
    \mathbf{x}_{k+1} = \mathbf{f} \left( \mathbf{x}_k \right) + \mathbf{\Gamma}_{k+1|k} \mathbf{w}_k
\end{equation*}

where $\mathbf{x}$ is an $n \times 1$ state vector, $\mathbf{f}()$ is an $n \times 1$
matrix function, $\mathbf{\Gamma}$ is an $n \times p$ transformation matrix, and $\mathbf{w}_k$
is a $p \times 1$ white noise sequence with covariance

\begin{equation*}
    \mathbf{Q}_k = E \left\{ \mathbf{w}_k \mathbf{w}_k^T \right\}
\end{equation*}

and the nonlinear observation model

\begin{equation*}
    \mathbf{z}_k = \mathbf{h} \left( \mathbf{x}_k \right) + \mathbf{v}_k
\end{equation*}

where $\mathbf{z}$ is an $m \times 1$ vector, $\mathbf{h}()$ is an $m \times 1$ matrix
function, and $\mathbf{v}_k$ is an $m \times 1$ white noise sequence with covariance

\begin{equation*}
    \mathbf{R}_k = E \left\{ \mathbf{v}_k \mathbf{v}_k^T \right\}
\end{equation*}

The process and observation noise sequences are uncorrelated

\begin{equation*}
    E \left\{ \mathbf{w}_j \mathbf{v}_k^T \right\} = \mathbf{0} \, , \phantom{.} \mathrm{for} \, \mathrm{all} \, j \, \mathrm{and} \, k
\end{equation*}

Recall that the extended Kalman filter in \S\ref{The Extended Kalman Filter} approximates
the nonlinear transformations with first-order Taylor coefficient matrices. There are
situations where these approximations can be inadequate in sufficiently modeling the
system dynamics and observation models, impacting the quality of the Kalman filter
performance. The unscented Kalman filter attempts to address these shortcomings by
relying on the notion that
\textit{it is better to approximate a probability distribution than to approximate a nonlinear transformation}.
The basic approach is to use the standard deviations derived from the square root of the
variances of the state estimation error covariance, $\mathbf{P}$, to generate a set of
"sigma points" that approximate the probability distributions of the process and
observation operations. These sigma points are then used in weighted sum calculations
to generate the necessary quantities needed for the Kalman filter update operations.

We begin with the set of $2n + 1$ weighting factors $\left\{ W^{(0)}, W^{(1)}, ..., W^{(2n)} \right\}$,
where, in order to provide unbiased estimates, the weighting factors must sum to unity

\begin{equation*}
    \sum_{j=0}^{2n} W^{(j)} = 1
\end{equation*}

We can choose $W^{(0)}$ based on an empirical criteria. The remaining $2n$ weights are
then obtained from

\begin{equation*}
    W^{(j)} = \frac{1 - W^{(0)}}{2 n} \; , j=1,\dots,2n
\end{equation*}

The $2n + 1$ sigma points, $\mathbf{X}^{(j)}$, for $\hat{\mathbf{x}}_{k-1}$ and $\mathbf{P}_{k-1}$
are determined from:

\begin{equation*}
    \begin{aligned}
        \mathbf{L}_{k-1} &= \sqrt{ \frac{n}{1 - W^{(0)}} \, \mathbf{P}_{k-1} } \\
        \phantom{X} \\
        \mathbf{X}^{(0)} &= \hat{\mathbf{x}}_{k-1} \\
        \mathbf{X}^{(i)} &= \hat{\mathbf{x}}_{k-1} + \left[ \mathbf{L}_{k-1} \right]_i \; , i=1,\dots,n \\
        \mathbf{X}^{(i+n)} &= \hat{\mathbf{x}}_{k-1} - \left[ \mathbf{L}_{k-1} \right]_i \; , i=1,\dots,n
    \end{aligned}
\end{equation*}

where $\sqrt{\mathbf{M}}$ is the lower triangular Cholesky decomposition of the symmetric
matrix $\mathbf{M}$, and $\left[ \mathbf{L} \right]_i$ is the $i^{\, \text{th}}$ column
of the lower triangular matrix $\mathbf{L}$. Basically, $\left[ \mathbf{L} \right]_i$
represents the estimation error standard deviation matrix of the $i^{\, \text{th}}$
component of $\hat{\mathbf{x}}_{k-1}$.

The primary weight, $W^{(0)}$, controls the spread of the sigma points. $W^{(0)} < 0$ moves
the sigma points closer to the distribution origin, whereas $W^{(0)} > 0$ moves the sigma
points further from the distribution origin.

One suggested approach by Julier and Uhlmann \cite{julieruhlmann1997} for choosing the
weights is to set $W^{(0)}$ to

\begin{equation*}
    W^{(0)} = \frac{\kappa}{n + \kappa}
\end{equation*}

Then the remaining $2n$ weights become

\begin{equation*}
    W^{(j)} = \frac{1}{2 (n + \kappa)} \; , j=1,\dots,2n
\end{equation*}

and $\mathbf{L}_{k-1}$ becomes

\begin{equation*}
    \mathbf{L}_{k-1} = \sqrt{ (n + \kappa) \, \mathbf{P}_{k-1} }
\end{equation*}

The value $\kappa$ is used to "fine tune" the higher order moments of the distribution
approximation. When the distribution is assumed to be Gaussian, then $\kappa$ should be
chosen so that

\begin{equation*}
    n + \kappa = 3
\end{equation*}

Given a set of $2n + 1$ weighting factors, $W^{(j)}$, and a set of $2n + 1$ sigma points,
$\mathbf{X}^{(j)}$, the unscented Kalman filter update cycle for a nonlinear system is

I. Projection (\textit{a priori}) update:

\begin{equation*}
    \begin{aligned}
        \mathbf{x}^{(j)} &= \mathbf{f} \left( \mathbf{X}^{(j)} \right) \; , j=0,\dots,2n \\
        \hat{\mathbf{x}}_{k|k-1} &= \sum_{j=0}^{2n} W^{(j)} \mathbf{x}^{(j)} \\
        \phantom{X} \\
        \mathbf{P}_{k|k-1} &= \sum_{j=0}^{2n} W^{(j)} \left[ \mathbf{x}^{(j)} - \hat{\mathbf{x}}_{k|k-1} \right] \left[ \mathbf{x}^{(j)} - \hat{\mathbf{x}}_{k|k-1} \right]^T
        + \mathbf{\Gamma}_{k|k-1} \mathbf{Q}_{k-1} \mathbf{\Gamma}_{k|k-1}^T
    \end{aligned}
\end{equation*}

II. Correction (\textit{a posteriori}) update:

\begin{equation*}
    \begin{aligned}
        \mathbf{z}^{(j)} &= \mathbf{h} \left( \mathbf{x}^{(j)} \right) \; , j=0,\dots,2n \\
        \hat{\mathbf{z}}_k &= \sum_{j=0}^{2n} W^{(j)} \mathbf{z}^{(j)} \\
        \phantom{X} \\
        \tilde{\mathbf{z}}_k &= \mathbf{z}_k - \hat{\mathbf{z}}_k \\
        \phantom{X} \\
        \mathbf{P}_{\mathbf{zz},k} &= \sum_{j=0}^{2n} W^{(j)} \left[ \mathbf{z}^{(j)} - \hat{\mathbf{z}}_k \right] \left[ \mathbf{z}^{(j)} - \hat{\mathbf{z}}_k \right]^T \\
        \mathbf{P}_{\mathbf{xz},k} &= \sum_{j=0}^{2n} W^{(j)} \left[ \mathbf{x}^{(j)} - \hat{\mathbf{x}}_{k|k-1} \right] \left[ \mathbf{z}^{(j)} - \hat{\mathbf{z}}_k \right]^T \\
        \phantom{X} \\
        \mathbf{S}_{k} &= \mathbf{P}_{\mathbf{zz},k} + \mathbf{R}_{k} \\
        \mathbf{K}_{k} &= \mathbf{P}_{\mathbf{xz},k} \; \mathbf{S}_{k}^{-1} \\
        \phantom{X} \\
        \hat{\mathbf{x}}_{k} &= \hat{\mathbf{x}}_{k|k-1} +\mathbf{K}_k \, \tilde{\mathbf{z}}_k \\
        \mathbf{P}_{k} &= \mathbf{P}_{k|k-1} - \mathbf{K}_k \, \mathbf{S}_k \, \mathbf{K}_k^T
    \end{aligned}
\end{equation*}


