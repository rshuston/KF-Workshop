\section{Conventions and Notations}
\label{Conventions and Notations}

When I began my journey into studying the Kalman filter, my primary reference was the
"Robert Grover Brown" book \cite{rgbrown1983}, and much of my choice of conventions and
notations comes from that book. I also heavily relied on Sorenson’s material
\cite{sorenson1985} which was consistent with R.G. Brown notation. As I accumulated other
references over the years, I discovered that the Brown/Sorenson conventions and notations
were dominant. Granted there are other conventions and notations in the Kalman filtering
literature, but I feel most at home when using the Brown/Sorenson conventions and notations,
and those are the ones I’ve chosen to adopt in this document.

In particular, $\mathbf{x}_k$ represents a system state vector at time event "$k$",
$\mathbf{\Phi}_{k+1|k}$ represents a linear state transition matrix that transitions the
state from time events "$k$" to "$k+1$",
$\mathbf{z}_k$ represents an observation (i.e., measurement) vector that observes the
system output at time event "$k$",
and $\mathbf{H}_k$ represents a linear observation transformation from $\mathbf{x}_k$
to $\mathbf{z}_k$.
Additionally, $\mathbf{w}_k$ and $\mathbf{v}_k$ represent random sequences that model the
system dynamics variations and observation variations, respectively, with covariances
$\mathbf{Q}_k$ and $\mathbf{R}_k$, respectively.
Lastly, the system state estimation error covariance is represented by $\mathbf{P}_k$.

In addition to the many different choices of variables in the technical literature, there
are also many notational styles.
For instance, instead of using the "$k$" and "$k+1|k$" subscripts, some references use
functional forms instead, as in "$\mathbf{x}(k)$" and "$\mathbf{\Phi}(k+1|k)$".
Others use "$+$" and "$-$" indicators to differentiate \textit{a priori} variables from
\textit{a posteriori} variables, e.g., $\mathbf{P}_k^-$ instead of $\mathbf{P}_{k|k-1}$,
and $\mathbf{P}_k^+$ instead of $\mathbf{P}_{k|k}$.
The notations adopted in this document strive to make the equations clear (or as clear
as they can be) without being "too busy" in the variable descriptors;
$\mathbf{\Phi}_{k+1|k}$ is more readable in an expression than $\mathbf{\Phi}(k+1|k)$.

There are many references that do not distinguish visually between a scalar quantity and
a matrix quantity. While this may have been a necessity in the days of typewriters and
early word processors, it also does the reader an extreme disservice. Knowing that a
variable is a scalar or a matrix is fundamental in the understanding of an equation or
derivation, and given today’s technology in word processing and electronic typesetting,
there is no excuse for not adhering to the accepted conventions where an italic variable,
$s$, is a scalar quantity, and a boldface variable, $\mathbf{M}$, is a matrix quantity.

I’ve often wondered why this particular choice of variables was picked by certain authors.
For instance, why use $\mathbf{z}$ for the observation vector when the state vector is
$\mathbf{x}$ and the natural choice would be to use $\mathbf{y}$ instead? After all,
that’s what all the classical linear control system theory resources use. And why use
$\mathbf{H}$ as the linear observation transformation instead of something more consistent
with classical linear control system theory? My uneducated guess is that using $\mathbf{z}$
allows for the use of $\mathbf{y}$ to represent additional state values (e.g., hidden
states) in augmented system formulations, and using $\mathbf{H}$ is simply a natural
alphabetic progression after $\mathbf{F}$ and $\mathbf{G}$ are assigned for discrete-time
system functional transformations. For instance, consider the classical continuous-time
linear system description:

\begin{equation*}
    \begin{aligned}
        \mathbf{\dot{x}} &= \mathbf{A} \mathbf{x} + \mathbf{B} \mathbf{u} \\
        \mathbf{y} &= \mathbf{C} \mathbf{x} + \mathbf{D} \mathbf{u} \\
    \end{aligned}
\end{equation*}

where $\mathbf{x}$ is the system state, $\mathbf{u}$ is the control input, and $\mathbf{y}$
is the output observation. In transforming to a discrete-time description, a natural
choice of variables would be

\begin{equation*}
    \begin{aligned}
        \mathbf{x}_{k+1} &= \mathbf{F}_k \mathbf{x}_k + \mathbf{G}_k \mathbf{u}_k \\
        \mathbf{z}_k &= \mathbf{H}_k \mathbf{x}_k + \mathbf{J}_k \mathbf{u}_k
    \end{aligned}
\end{equation*}

We should stay away from using $\mathbf{E}$ because it could easily be confused with the
expectation operation $E \left\{ \cdots \right\}$, and we can’t use $\mathbf{I}$ because
that’s the identity matrix (although I’ve seen some authors use $\mathbf{I}_k$ to represent
an information matrix value at time event "$k$", causing me horrible confusion), so the
choices of $\mathbf{H}$ seems to make logical sense. Similarly, the choices for covariance
values appear to be alphabetically assigned. If $\mathbf{P}$ is our first-assigned covariance,
then using $\mathbf{Q}$ and $\mathbf{R}$ make sense for the next two assignments. Of course,
this reasoning behind the assignments is pure conjecture on my part, but it allows me to
find order in the conventions that I’ve chosen from R.G. Brown and others.

The following list summarizes the general conventions and notations used in this document:

\begin{myitemize}
    \item An italic variable, $s$, designates a scalar quantity (the variable can be of any case)
    \item A boldface lower-case variable, $\mathbf{v}$, designates a vector matrix quantity
    \item A boldface upper-case variable, $\mathbf{M}$, designates a block matrix quantity
    \item $\mathbf{x}_k$ designates a system state vector at time event "$k$"
    \item $\hat{\mathbf{x}}_k$ designates an estimate of the system state vector, $\mathbf{x}_k$, at time event "$k$"
    \item $\mathbf{P}_k$ designates the covariance matrix of the state estimation error, $\mathbf{x}_k - \hat{\mathbf{x}}_k$
    \item $\mathbf{\Phi}_{k+1|k}$ designates a linear state transition matrix that transitions $\mathbf{x}_k$ from time events "$k$" to "$k+1$"
    \item $\mathbf{u}_k$ designates a system control input vector at time event "$k$"
    \item $\mathbf{w}_k$ designates a system process random variation vector at time event "$k$"
    \item $\mathbf{Q}_k$ designates the covariance matrix of the system process random variation, $\mathbf{w}_k$
    \item $\mathbf{\Gamma}_{k+1|k}$ designates a linear matrix that transforms $\mathbf{w}_k$ from time events "$k$" to "$k+1$" into the state space of $\mathbf{x}_k$
    \item $\mathbf{z}_k$ designates an observation (i.e., measurement) vector that observes the system at time event "$k$"
    \item $\hat{\mathbf{z}}_k$ designates an estimate of the observation vector, $\mathbf{z}_k$, at time event "$k$"
    \item $\mathbf{R}_k$ designates the covariance matrix of the observation estimation error, $\mathbf{z}_k - \hat{\mathbf{z}}_k$
    \item $\mathbf{H}_k$ designates a linear observation transformation matrix from the state space of $\mathbf{x}_k$ into the state space of $\mathbf{z}_k$
    \item $\mathbf{v}_k$ designates an observation random variation vector at time event "$k$"
    \item $\mathbf{R}_k$ designates the covariance matrix of the observation random variation, $\mathbf{v}_k$
    \item $\mathbf{K}_k$ designates the state estimation filter gain matrix, i.e., the Kalman gain
\end{myitemize}
