\section{Quadratic Factors}
\label{Quadratic Factors}

One of the critical computation objectives is the preservation of the symmetric and
positive definite characteristics of the covariance matrices. In particular, there are
three matrices of interest: $\mathbf{P}_{k|k-1}$, $\mathbf{S}_k$, and $\mathbf{P}_k$.
Recalling the expressions for each of these from
(\ref{eq:a-priori-state-covariance}),
(\ref{eq:observation-residual-covariance}),
and (\ref{eq:a-posteriori-state-covariance-quadratic}),
respectively, we have

\begin{equation*}
    \begin{aligned}
        & \mathbf{P}_{k|k-1} = \mathbf{\Phi}_{k|k-1} \, \mathbf{P}_{k-1} \, \mathbf{\Phi}_{k|k-1}^T + \mathbf{\Gamma}_{k|k-1} \mathbf{Q}_{k-1} \mathbf{\Gamma}_{k|k-1}^T \\
        & \mathbf{S}_{k} = \mathbf{H}_{k} \, \mathbf{P}_{k|k-1} \, \mathbf{H}_{k}^T + \mathbf{R}_{k} \\
        & \mathbf{P}_k = \mathbf{P}_{k|k-1} - \mathbf{K}_{k} \, \mathbf{S}_{k} \, \mathbf{K}_{k}^T
    \end{aligned}
\end{equation*}

Observe that each of these contains one or more quadratic terms of the form $\mathbf{C} = \mathbf{A} \mathbf{B} \mathbf{A}^T$,
where $\mathbf{B}$ is a square, symmetric matrix. Hence, $\mathbf{C}$ is also a square,
symmetric matrix. Let $\mathbf{A}$ be of dimension $m \times n$ and $\mathbf{B}$ be of
dimension $n \times n$. Then $\mathbf{C}$ will be of dimension $m \times m$.

Algorithm \ref{alg:QuadraticMultiply} below gives a procedure to compute
computing $\mathbf{C} = \mathbf{A} \mathbf{B} \mathbf{A}^T$ while preserving symmetry.

\begin{algorithm}
    \caption{Quadratic Multiply}
    \label{alg:QuadraticMultiply}
    \begin{algorithmic}
        \FOR{$i = 1$ \textbf{to} $m$}
            \STATE $\sigma = 0$
            \FOR{$j = 1$ \textbf{to} $n$}
                \FOR{$k = 1$ \textbf{to} $n$}
                    \STATE $\sigma = \sigma + \mathbf{A}_{ij} * \mathbf{B}_{jk} * \mathbf{A}_{ik}$
                \ENDFOR
            \ENDFOR
            \STATE $\mathbf{C}_{ii} = \sigma$
            \FOR{$l = i+1$ \textbf{to} $m$}
                \STATE $\sigma = 0$
                \FOR{$j = 1$ \textbf{to} $n$}
                    \FOR{$k = 1$ \textbf{to} $n$}
                        \STATE $\sigma = \sigma + \mathbf{A}_{ij} * \mathbf{B}_{jk} * \mathbf{A}_{lk}$
                    \ENDFOR
                \ENDFOR
                \STATE $\mathbf{C}_{il} = \sigma$
                \STATE $\mathbf{C}_{li} = \sigma$
            \ENDFOR
        \ENDFOR
    \end{algorithmic}
\end{algorithm}

