\section{The Information Form of the Kalman Filter}
\label{The Information Form of the Kalman Filter}

The inverse of a covariance matrix is known as an information matrix. Whereas a
covariance matrix is an indicator of dispersion, an information matrix is an indicator
of precision. The smaller the covariance is, the higher the precision is, and vice versa.
An information matrix of $\mathbf{0}$ value implies infinite covariance, meaning that no
statistical information can be inferred. The Kalman filter can be reformulated to use the
inverse of $\mathbf{P}$, although this form of the Kalman filter is rarely used for
implementation. However, it does have useful theoretical purposes.

Recall the Kalman filter \textit{a posteriori} covariance expression given in
(\ref{eq:a-posteriori-state-covariance-quadratic})

\begin{equation*}
    \mathbf{P}_k = \mathbf{P}_{k|k-1} - \mathbf{K}_{k} \, \mathbf{S}_{k} \, \mathbf{K}_{k}^T
\end{equation*}

Remembering that $\mathbf{P}$ and $\mathbf{S}$ are symmetric, inserting the expression
for $\mathbf{K}_k$ gives

\begin{equation*}
    \begin{aligned}
        \mathbf{P}_k
        &= \mathbf{P}_{k|k-1} - \left( \mathbf{P}_{k|k-1} \mathbf{H}_k^T \mathbf{S}_k^{-1} \right) \mathbf{S}_{k} \left( \mathbf{P}_{k|k-1} \mathbf{H}_k^T \mathbf{S}_k^{-1} \right)^T \\
        &= \mathbf{P}_{k|k-1} - \mathbf{P}_{k|k-1} \mathbf{H}_k^T \mathbf{S}_k^{-1} \mathbf{S}_{k} \mathbf{S}_k^{-1} \mathbf{H}_k \mathbf{P}_{k|k-1} \\
        &= \mathbf{P}_{k|k-1} - \mathbf{P}_{k|k-1} \mathbf{H}_k^T \mathbf{S}_k^{-1} \mathbf{H}_k \mathbf{P}_{k|k-1}
    \end{aligned}
\end{equation*}

and then inserting the expression for $\mathbf{S}_k$ gives

\begin{equation}
    \mathbf{P}_k = \mathbf{P}_{k|k-1} - \mathbf{P}_{k|k-1} \mathbf{H}_k^T \left[ \mathbf{H}_k \mathbf{P}_{k|k-1} \mathbf{H}_k^T + \mathbf{R}_k \right]^{-1} \mathbf{H}_k \mathbf{P}_{k|k-1}
    \label{eq:state-covariance-quadratic-expanded}
\end{equation}

Our objective is to transform (\ref{eq:state-covariance-quadratic-expanded}) into a form
based on $\mathbf{P}_k^{-1}$ and $\mathbf{P}_{k|k-1}^{-1}$. Observe that this equation is of the form

\begin{equation}
    \mathbf{P}_k = \mathbf{A} - \mathbf{B}^T \mathbf{C} \mathbf{B}
    \label{eq:state-covariance-quadratic-form}
\end{equation}

where

\begin{equation*}
    \begin{aligned}
        \mathbf{A} &= \mathbf{P}_{k|k-1} \\
        \mathbf{B} &= \mathbf{H}_k \mathbf{P}_{k|k-1} \\
        \mathbf{B}^T & = \mathbf{P}_{k|k-1} \mathbf{H}_k^T \\
        \mathbf{C} &= \left[ \mathbf{H}_k \mathbf{P}_{k|k-1} \mathbf{H}_k^T + \mathbf{R}_k \right]^{-1}
    \end{aligned}
\end{equation*}

We now need to introduce some useful matrix lemmas \cite{brookes2020}.

\boxed{
\parbox{\textwidth}{
\begin{lemma}
\label{AB-matrix-expansion-identity}
Let $\mathbf{A}$ and $\mathbf{B}$ both be square matrices of equal dimensions.
Then.
\begin{equation*}
    \mathbf{A} \left( \mathbf{I} + \mathbf{B} \mathbf{A} \right) = \left( \mathbf{I} + \mathbf{A} \mathbf{B} \right) \mathbf{A}
\end{equation*}
\end{lemma}
}
}

\begin{proof}
\begin{equation*}
    \begin{aligned}
        \mathbf{A} \left( \mathbf{I} + \mathbf{B} \mathbf{A} \right) &= \mathbf{A} + \mathbf{A} \mathbf{B} \mathbf{A} \\
        &= \left( \mathbf{I} + \mathbf{A} \mathbf{B} \right) \mathbf{A}
    \end{aligned}
\end{equation*}
\end{proof}

\boxed{
\parbox{\textwidth}{
\begin{lemma}
\label{AB-matrix-inverse-expansion-identity}
Let $\mathbf{A}$ and $\mathbf{B}$ both be square and invertible matrices of equal dimensions.
Then.
\begin{equation*}
    \mathbf{A} \left( \mathbf{I} + \mathbf{B} \mathbf{A} \right)^{-1} = \left( \mathbf{I} + \mathbf{A} \mathbf{B} \right)^{-1} \mathbf{A}
\end{equation*}
\end{lemma}
}
}

\begin{proof}
From Lemma \ref{AB-matrix-expansion-identity}, we see that
\begin{equation*}
    \begin{aligned}
        \mathbf{A} \left( \mathbf{I} + \mathbf{B} \mathbf{A} \right)
        &= \left( \mathbf{I} + \mathbf{A} \mathbf{B} \right) \mathbf{A} \\
        \left[ \mathbf{A} \left( \mathbf{I} + \mathbf{B} \mathbf{A} \right) \right]^{-1}
        &= \left[ \left( \mathbf{I} + \mathbf{A} \mathbf{B} \right) \mathbf{A} \right]^{-1} \\
        \left( \mathbf{I} + \mathbf{B} \mathbf{A} \right)^{-1} \mathbf{A}^{-1}
        &= \mathbf{A}^{-1} \left( \mathbf{I} + \mathbf{A} \mathbf{B} \right)^{-1} \\
        \mathbf{A} \left( \mathbf{I} + \mathbf{B} \mathbf{A} \right)^{-1} \mathbf{A}^{-1}
        &= \left( \mathbf{I} + \mathbf{A} \mathbf{B} \right)^{-1} \\
        \mathbf{A} \left( \mathbf{I} + \mathbf{B} \mathbf{A} \right)^{-1}
        &= \left( \mathbf{I} + \mathbf{A} \mathbf{B} \right)^{-1} \mathbf{A}
    \end{aligned}
\end{equation*}
\end{proof}

\boxed{
\parbox{\textwidth}{
\begin{lemma}
\label{A-matrix-inverse-expansion-identity}
Let $\mathbf{A}$ be a square and invertible matrix.
Then.
\begin{equation*}
    \left( \mathbf{I} + \mathbf{A} \right)^{-1} = \mathbf{I} - \left( \mathbf{I} + \mathbf{A} \right)^{-1} \mathbf{A}
\end{equation*}
\end{lemma}
}
}

\begin{proof}
Noting that multiplying by $\mathbf{I} + \mathbf{A} - \mathbf{A}$ does not alter the value
of $\left( \mathbf{I} + \mathbf{A} \right)^{-1}$, we see that
\begin{equation*}
    \begin{aligned}
        \left( \mathbf{I} + \mathbf{A} \right)^{-1}
        &= \left( \mathbf{I} + \mathbf{A} \right)^{-1} \left( \mathbf{I} + \mathbf{A} - \mathbf{A} \right) \\
        &= \left( \mathbf{I} + \mathbf{A} \right)^{-1} \left( \mathbf{I} + \mathbf{A} \right)
        - \left( \mathbf{I} + \mathbf{A} \right)^{-1} \left( \mathbf{A} \right) \\
        &= \mathbf{I} - \left( \mathbf{I} + \mathbf{A} \right)^{-1} \mathbf{A}
    \end{aligned}
\end{equation*}
\end{proof}

\boxed{
\parbox{\textwidth}{
\begin{lemma}
\label{ABCD-matrix-inverse-expansion-identity}
Let $\mathbf{A}$ be a square and invertible matrix, and let $\mathbf{B}$, $\mathbf{C}$,
and $\mathbf{D}$ be matrices such that the expression $\mathbf{A} + \mathbf{B} \mathbf{C} \mathbf{D}$
is invertible.
Then.
\begin{equation*}
    \left( \mathbf{A} + \mathbf{B} \mathbf{C} \mathbf{D} \right)^{-1} =
    \mathbf{A}^{-1} - \left( \mathbf{I} + \mathbf{A}^{-1}\mathbf{B} \mathbf{C} \mathbf{D} \right)^{-1} \mathbf{A}^{-1} \mathbf{B} \mathbf{C} \mathbf{D} \mathbf{A}^{-1}
\end{equation*}
\end{lemma}
}
}

\begin{proof}
Using Lemma \ref{A-matrix-inverse-expansion-identity}, we have that
\begin{equation*}
    \begin{aligned}
        \left( \mathbf{A} + \mathbf{B} \mathbf{C} \mathbf{D} \right)^{-1}
        &= \left( \mathbf{A} + \mathbf{A} \mathbf{A}^{-1}\mathbf{B} \mathbf{C} \mathbf{D} \right)^{-1} \\
        &= \left( \mathbf{A} \left[ \mathbf{I} + \mathbf{A}^{-1} \mathbf{B} \mathbf{C} \mathbf{D} \right] \right)^{-1} \\
        &= \left[ \mathbf{I} + \mathbf{A}^{-1} \mathbf{B} \mathbf{C} \mathbf{D} \right]^{-1} \mathbf{A}^{-1} \\
        &= \left[ \mathbf{I} - \left( \mathbf{I} + \mathbf{A}^{-1}\mathbf{B} \mathbf{C} \mathbf{D} \right)^{-1} \mathbf{A}^{-1} \mathbf{B} \mathbf{C} \mathbf{D} \right] \mathbf{A}^{-1} \\
        &= \mathbf{A}^{-1} - \left( \mathbf{I} + \mathbf{A}^{-1}\mathbf{B} \mathbf{C} \mathbf{D} \right)^{-1} \mathbf{A}^{-1} \mathbf{B} \mathbf{C} \mathbf{D} \mathbf{A}^{-1}
    \end{aligned}
\end{equation*}
\end{proof}

\boxed{
\parbox{\textwidth}{
\begin{lemma}
\label{ABCD-matrix-inverse-expansion-identity-three}
Let $\mathbf{A}$ be a square and invertible matrix, and let $\mathbf{B}$, $\mathbf{C}$,
and $\mathbf{D}$ be matrices such that the expression $\mathbf{A} + \mathbf{B} \mathbf{C} \mathbf{D}$
is invertible.
Then.
\begin{equation*}
    \left( \mathbf{A} + \mathbf{B} \mathbf{C} \mathbf{D} \right)^{-1} =
    \mathbf{A}^{-1} - \mathbf{A}^{-1} \mathbf{B} \left( \mathbf{I} + \mathbf{C} \mathbf{D} \mathbf{A}^{-1} \mathbf{B} \right)^{-1} \mathbf{C} \mathbf{D} \mathbf{A}^{-1}
\end{equation*}
\end{lemma}
}
}

\begin{proof}
By repeatedly applying Lemmas \ref{AB-matrix-inverse-expansion-identity}
and \ref{A-matrix-inverse-expansion-identity}
to Lemma \ref{ABCD-matrix-inverse-expansion-identity}, we have the following equivalence relations
\begin{equation*}
    \begin{aligned}
        \left( \mathbf{A} + \mathbf{B} \mathbf{C} \mathbf{D} \right)^{-1}
        &= \mathbf{A}^{-1} - \left( \mathbf{I} + \mathbf{A}^{-1}\mathbf{B} \mathbf{C} \mathbf{D} \right)^{-1} \mathbf{A}^{-1} \mathbf{B} \mathbf{C} \mathbf{D} \mathbf{A}^{-1}
        & (1) \\
        &= \mathbf{A}^{-1} - \mathbf{A}^{-1} \left( \mathbf{I} + \mathbf{B} \mathbf{C} \mathbf{D} \mathbf{A}^{-1} \right)^{-1} \mathbf{B} \mathbf{C} \mathbf{D} \mathbf{A}^{-1}
        & (2) \\
        &= \mathbf{A}^{-1} - \mathbf{A}^{-1} \mathbf{B} \left( \mathbf{I} + \mathbf{C} \mathbf{D} \mathbf{A}^{-1} \mathbf{B} \right)^{-1} \mathbf{C} \mathbf{D} \mathbf{A}^{-1}
        & (3) \\
        &= \mathbf{A}^{-1} - \mathbf{A}^{-1} \mathbf{B} \mathbf{C} \left( \mathbf{I} + \mathbf{D} \mathbf{A}^{-1} \mathbf{B} \mathbf{C} \right)^{-1} \mathbf{D} \mathbf{A}^{-1}
        & (4) \\
        &= \mathbf{A}^{-1} - \mathbf{A}^{-1} \mathbf{B} \mathbf{C} \mathbf{D} \left( \mathbf{I} + \mathbf{A}^{-1} \mathbf{B} \mathbf{C} \mathbf{D} \right)^{-1} \mathbf{A}^{-1}
        & (5) \\
        &= \mathbf{A}^{-1} - \mathbf{A}^{-1} \mathbf{B} \mathbf{C} \mathbf{D} \mathbf{A}^{-1} \left( \mathbf{I} + \mathbf{B} \mathbf{C} \mathbf{D} \mathbf{A}^{-1} \right)^{-1}
        & (6)
    \end{aligned}
\end{equation*}
The third equivalence relation establishes the lemma.
\end{proof}

\boxed{
\parbox{\textwidth}{
\begin{lemma}
\label{the-matrix-inversion-lemma}
Let $\mathbf{A}$ and $\mathbf{C}$ both be square and invertible matrices, and let $\mathbf{B}$, $\mathbf{C}$,
and $\mathbf{D}$ be matrices such that the expression $\mathbf{A} + \mathbf{B} \mathbf{C} \mathbf{D}$
is invertible.
Then.
\begin{equation*}
    \left( \mathbf{A} + \mathbf{B} \mathbf{C} \mathbf{D} \right)^{-1}
    = \mathbf{A}^{-1} - \mathbf{A}^{-1} \mathbf{B} \left( \mathbf{C}^{-1} + \mathbf{D} \mathbf{A}^{-1} \mathbf{B} \right)^{-1} \mathbf{D} \mathbf{A}^{-1}
\end{equation*}
\end{lemma}
}
}

\begin{proof}
Because $\mathbf{C}^{-1}$ exists, then Lemma \ref{ABCD-matrix-inverse-expansion-identity-three} becomes
\begin{equation*}
    \begin{aligned}
        \left( \mathbf{A} + \mathbf{B} \mathbf{C} \mathbf{D} \right)^{-1}
        &= \mathbf{A}^{-1} - \mathbf{A}^{-1} \mathbf{B} \left( \mathbf{I} + \mathbf{C} \mathbf{D} \mathbf{A}^{-1} \mathbf{B} \right)^{-1} \mathbf{C} \mathbf{D} \mathbf{A}^{-1} \\
        &= \mathbf{A}^{-1} - \mathbf{A}^{-1} \mathbf{B} \left( \mathbf{C} \mathbf{C}^{-1} + \mathbf{C} \mathbf{D} \mathbf{A}^{-1} \mathbf{B} \right)^{-1} \mathbf{C} \mathbf{D} \mathbf{A}^{-1} \\
        &= \mathbf{A}^{-1} - \mathbf{A}^{-1} \mathbf{B} \left( \mathbf{C} \left[ \mathbf{C}^{-1} + \mathbf{D} \mathbf{A}^{-1} \mathbf{B} \right] \right)^{-1} \mathbf{C} \mathbf{D} \mathbf{A}^{-1} \\
        &= \mathbf{A}^{-1} - \mathbf{A}^{-1} \mathbf{B} \left( \mathbf{C}^{-1} + \mathbf{D} \mathbf{A}^{-1} \mathbf{B} \right)^{-1} \mathbf{C}^{-1} \mathbf{C} \mathbf{D} \mathbf{A}^{-1} \\
        &= \mathbf{A}^{-1} - \mathbf{A}^{-1} \mathbf{B} \left( \mathbf{C}^{-1} + \mathbf{D} \mathbf{A}^{-1} \mathbf{B} \right)^{-1} \mathbf{D} \mathbf{A}^{-1}
    \end{aligned}
\end{equation*}
\end{proof}

It is worth mentioning that Lemma \ref{the-matrix-inversion-lemma} is a commonly
referenced matrix identity in many applications, for example, linear quadratic regulator
control theory. Is is commonly titled \textit{"The Matrix Inversion Lemma"} the technical literature.

\boxed{
\parbox{\textwidth}{
\begin{lemma}
\label{matrix-inversion-lemma-negative}
Let $\mathbf{A}$ and $\mathbf{C}$ both be square and invertible matrices, and let $\mathbf{B}$, $\mathbf{C}$,
and $\mathbf{D}$ be matrices such that the expression $\mathbf{A} - \mathbf{B} \mathbf{C} \mathbf{D}$
is invertible.
Then.
\begin{equation*}
    \left( \mathbf{A} - \mathbf{B} \mathbf{C} \mathbf{D} \right)^{-1}
    = \mathbf{A}^{-1} + \mathbf{A}^{-1} \mathbf{B} \left( \mathbf{C}^{-1} - \mathbf{D} \mathbf{A}^{-1} \mathbf{B} \right)^{-1} \mathbf{D} \mathbf{A}^{-1}
\end{equation*}
\end{lemma}
}
}

\begin{proof}
Let $\mathbf{F}$ be a matrix compatible with Lemma \ref{the-matrix-inversion-lemma} so that
\begin{equation*}
    \left( \mathbf{A} + \mathbf{F} \mathbf{C} \mathbf{D} \right)^{-1}
    = \mathbf{A}^{-1} - \mathbf{A}^{-1} \mathbf{F} \left( \mathbf{C}^{-1} + \mathbf{D} \mathbf{A}^{-1} \mathbf{F} \right)^{-1} \mathbf{D} \mathbf{A}^{-1}
\end{equation*}
Letting $\mathbf{F} = - \mathbf{B}$ we see that
\begin{equation*}
    \left( \mathbf{A} - \mathbf{B} \mathbf{C} \mathbf{D} \right)^{-1}
    = \mathbf{A}^{-1} + \mathbf{A}^{-1} \mathbf{B} \left( \mathbf{C}^{-1} - \mathbf{D} \mathbf{A}^{-1} \mathbf{B} \right)^{-1} \mathbf{D} \mathbf{A}^{-1}
\end{equation*}
\end{proof}

\boxed{
\parbox{\textwidth}{
\begin{lemma}
\label{matrix-inversion-lemma-A-BtCB}
Let $\mathbf{A}$ and $\mathbf{C}$ both be square and invertible matrices, and let $\mathbf{B}$ and $\mathbf{C}$
be matrices such that the expression $\mathbf{A} - \mathbf{B}^T \mathbf{C} \mathbf{B}$
is invertible.
Then.
\begin{equation*}
    \left( \mathbf{A} - \mathbf{B}^T \mathbf{C} \mathbf{B} \right)^{-1}
    = \mathbf{A}^{-1} + \mathbf{A}^{-1} \mathbf{B}^T \left( \mathbf{C}^{-1} - \mathbf{B} \mathbf{A}^{-1} \mathbf{B}^T \right)^{-1} \mathbf{B} \mathbf{A}^{-1}
\end{equation*}
\end{lemma}
}
}

\begin{proof}
Let $\mathbf{F}$ and $\mathbf{G}$ be matrices compatible with Lemma \ref{matrix-inversion-lemma-negative} so that
\begin{equation*}
    \left( \mathbf{A} - \mathbf{F} \mathbf{C} \mathbf{G} \right)^{-1}
    = \mathbf{A}^{-1} + \mathbf{A}^{-1} \mathbf{F} \left( \mathbf{C}^{-1} - \mathbf{G} \mathbf{A}^{-1} \mathbf{F} \right)^{-1} \mathbf{G} \mathbf{A}^{-1}
\end{equation*}
Letting $\mathbf{F} = \mathbf{B}^T$ and $\mathbf{G} = \mathbf{B}$ we see that
\begin{equation*}
    \left( \mathbf{A} - \mathbf{B}^T \mathbf{C} \mathbf{B} \right)^{-1}
    = \mathbf{A}^{-1} + \mathbf{A}^{-1} \mathbf{B}^T \left( \mathbf{C}^{-1} - \mathbf{B} \mathbf{A}^{-1} \mathbf{B}^T \right)^{-1} \mathbf{B} \mathbf{A}^{-1}
\end{equation*}
\end{proof}

Observe that we now have a Lemma expression where the term on the LHS,
$\mathbf{A} - \mathbf{B}^T \mathbf{C} \mathbf{B}$,
is exactly the same form as the RHS of (\ref{eq:state-covariance-quadratic-form})

\begin{equation*}
    \mathbf{P}_k = \mathbf{A} - \mathbf{B}^T \mathbf{C} \mathbf{B}
\end{equation*}

Recall that this is a structural representation of the expanded \textit{a posteriori}
covariance expression (\ref{eq:state-covariance-quadratic-expanded})

\begin{equation*}
    \mathbf{P}_k = \mathbf{P}_{k|k-1} - \mathbf{P}_{k|k-1} \mathbf{H}_k^T \left[ \mathbf{H}_k \mathbf{P}_{k|k-1} \mathbf{H}_k^T + \mathbf{R}_k \right]^{-1} \mathbf{H}_k \mathbf{P}_{k|k-1}
\end{equation*}

with

\begin{equation*}
    \begin{aligned}
        \mathbf{A} &= \mathbf{P}_{k|k-1} \\
        \mathbf{B} &= \mathbf{H}_k \mathbf{P}_{k|k-1} \\
        \mathbf{C} &= \left[ \mathbf{H}_k \mathbf{P}_{k|k-1} \mathbf{H}_k^T + \mathbf{R}_k \right]^{-1}
    \end{aligned}
\end{equation*}

so that

\begin{equation*}
    \begin{aligned}
        \mathbf{B}^T & = \mathbf{P}_{k|k-1} \mathbf{H}_k^T \\
        \mathbf{C}^{-1} &= \left[ \mathbf{H}_k \mathbf{P}_{k|k-1} \mathbf{H}_k^T + \mathbf{R}_k \right]
    \end{aligned}
\end{equation*}

Then, by applying Lemma \ref{matrix-inversion-lemma-A-BtCB}, we have
\begin{equation*}
    \begin{aligned}
        \mathbf{P}_k^{-1}
        &= \left( \mathbf{A} - \mathbf{B}^T \mathbf{C} \mathbf{B} \right)^{-1} \\
        &= \mathbf{A}^{-1} + \mathbf{A}^{-1} \mathbf{B}^T \left( \mathbf{C}^{-1} - \mathbf{B} \mathbf{A}^{-1} \mathbf{B}^T \right)^{-1} \mathbf{B} \mathbf{A}^{-1} \\
        &= \mathbf{P}_{k|k-1}^{-1} + \mathbf{P}_{k|k-1}^{-1} \mathbf{P}_{k|k-1} \mathbf{H}_k^T
        \left( \left[ \mathbf{H}_k \mathbf{P}_{k|k-1} \mathbf{H}_k^T + \mathbf{R}_k \right] \right. \\
        & \phantom{M} \left. - \mathbf{H}_k \mathbf{P}_{k|k-1} \mathbf{P}_{k|k-1}^{-1} \mathbf{P}_{k|k-1} \mathbf{H}_k^T \right)^{-1}
        \mathbf{H}_k \mathbf{P}_{k|k-1} \mathbf{P}_{k|k-1}^{-1} \\
        &= \mathbf{P}_{k|k-1}^{-1} + \mathbf{H}_k^T
        \left( \mathbf{H}_k \mathbf{P}_{k|k-1} \mathbf{H}_k^T + \mathbf{R}_k - \mathbf{H}_k \mathbf{P}_{k|k-1} \mathbf{H}_k^T \right)^{-1}
        \mathbf{H}_k \\
        &= \mathbf{P}_{k|k-1}^{-1} + \mathbf{H}_k^T \mathbf{R}_k^{-1} \mathbf{H}_k
    \end{aligned}
\end{equation*}

and so the \textit{a posteriori} correction update of the Kalman filter information matrix,
$\mathbf{P}_k^{-1}$, is

\boxed{
\parbox{\textwidth}{
\begin{equation}
    \mathbf{P}_k^{-1} = \mathbf{P}_{k|k-1}^{-1} + \mathbf{H}_k^T \mathbf{R}_k^{-1} \mathbf{H}_k
    \label{eq:state-correction-information-matrix}
\end{equation}
}
}

Next, recall the expression of the Kalman filter gain with $\mathbf{S}_k$ expanded

\begin{equation*}
    \begin{aligned}
        \mathbf{K}_k &= \mathbf{P}_{k|k-1} \mathbf{H}_k^T \mathbf{S}_k^{-1} \\
        &= \mathbf{P}_{k|k-1} \mathbf{H}_k^T \left[ \mathbf{H}_k \mathbf{P}_{k|k-1} \mathbf{H}_k^T + \mathbf{R}_k \right]^{-1}
    \end{aligned}
\end{equation*}

Strategically inserting $\mathbf{R}_k^{-1} \mathbf{R}_k$ does not alter the gain, and so

\begin{equation*}
    \begin{aligned}
        \mathbf{K}_k
        &= \mathbf{P}_{k|k-1} \mathbf{H}_k^T \left[ \mathbf{H}_k \mathbf{P}_{k|k-1} \mathbf{H}_k^T + \mathbf{R}_k \right]^{-1} \\
        &= \mathbf{P}_{k|k-1} \mathbf{H}_k^T \left( \mathbf{R}_k^{-1} \mathbf{R}_k \right) \left[ \mathbf{H}_k \mathbf{P}_{k|k-1} \mathbf{H}_k^T + \mathbf{R}_k \right]^{-1} \\
        &= \mathbf{P}_{k|k-1} \mathbf{H}_k^T \left( \mathbf{R}_k^{-1} \right) \left[ \left( \mathbf{H}_k \mathbf{P}_{k|k-1} \mathbf{H}_k^T + \mathbf{R}_k \right) \mathbf{R}_k^{-1} \right]^{-1} \\
        &= \mathbf{P}_{k|k-1} \mathbf{H}_k^T \mathbf{R}_k^{-1} \left[ \mathbf{H}_k \mathbf{P}_{k|k-1} \mathbf{H}_k^T \mathbf{R}_k^{-1} + \mathbf{I} \right]^{-1}
    \end{aligned}
\end{equation*}

Likewise, strategically inserting $\mathbf{P}_k \mathbf{P}_k^{-1}$ does not alter the gain, and so

\begin{equation*}
    \begin{aligned}
        \mathbf{K}_k
        &= \mathbf{P}_{k|k-1} \mathbf{H}_k^T \mathbf{R}_k^{-1} \left[ \mathbf{H}_k \mathbf{P}_{k|k-1} \mathbf{H}_k^T \mathbf{R}_k^{-1} + \mathbf{I} \right]^{-1} \\
        &= \left( \mathbf{P}_k \mathbf{P}_k^{-1} \right)
        \mathbf{P}_{k|k-1} \mathbf{H}_k^T \mathbf{R}_k^{-1} \left[ \mathbf{H}_k \mathbf{P}_{k|k-1} \mathbf{H}_k^T \mathbf{R}_k^{-1} + \mathbf{I} \right]^{-1} \\
        &= \mathbf{P}_k \left( \mathbf{P}_k^{-1} \right)
        \mathbf{P}_{k|k-1} \mathbf{H}_k^T \mathbf{R}_k^{-1} \left[ \mathbf{H}_k \mathbf{P}_{k|k-1} \mathbf{H}_k^T \mathbf{R}_k^{-1} + \mathbf{I} \right]^{-1} \\
        &= \mathbf{P}_k \left[ \mathbf{P}_{k|k-1}^{-1} + \mathbf{H}_k^T \mathbf{R}_k^{-1} \mathbf{H}_k \right]
        \mathbf{P}_{k|k-1} \mathbf{H}_k^T \mathbf{R}_k^{-1} \left[ \mathbf{H}_k \mathbf{P}_{k|k-1} \mathbf{H}_k^T \mathbf{R}_k^{-1} + \mathbf{I} \right]^{-1} \\
        &= \mathbf{P}_k \left[ \mathbf{I} + \mathbf{H}_k^T \mathbf{R}_k^{-1} \mathbf{H}_k \mathbf{P}_{k|k-1} \right]
        \mathbf{H}_k^T \mathbf{R}_k^{-1} \left[ \mathbf{H}_k \mathbf{P}_{k|k-1} \mathbf{H}_k^T \mathbf{R}_k^{-1} + \mathbf{I} \right]^{-1}
    \end{aligned}
\end{equation*}

Using Lemma \ref{AB-matrix-expansion-identity}, we see that the
$\left[ \mathbf{I} + \mathbf{H}_k^T \mathbf{R}_k^{-1} \mathbf{H}_k \mathbf{P}_{k|k-1} \right] \mathbf{H}_k^T \mathbf{R}_k^{-1}$
term becomes

\begin{equation*}
    \left[ \mathbf{I} + \mathbf{H}_k^T \mathbf{R}_k^{-1} \mathbf{H}_k \mathbf{P}_{k|k-1} \right] \mathbf{H}_k^T \mathbf{R}_k^{-1}
    = \mathbf{H}_k^T \mathbf{R}_k^{-1} \left[ \mathbf{I} + \mathbf{H}_k \mathbf{P}_{k|k-1} \mathbf{H}_k^T \mathbf{R}_k^{-1} \right]
\end{equation*}

and so

\begin{equation*}
    \begin{aligned}
        \mathbf{K}_k
        &= \mathbf{P}_k \left( \left[ \mathbf{I} + \mathbf{H}_k^T \mathbf{R}_k^{-1} \mathbf{H}_k \mathbf{P}_{k|k-1} \right]
        \mathbf{H}_k^T \mathbf{R}_k^{-1} \right) \left[ \mathbf{H}_k \mathbf{P}_{k|k-1} \mathbf{H}_k^T \mathbf{R}_k^{-1} + \mathbf{I} \right]^{-1} \\
        &= \mathbf{P}_k \left( \mathbf{H}_k^T \mathbf{R}_k^{-1} \left[ \mathbf{I} + \mathbf{H}_k \mathbf{P}_{k|k-1} \mathbf{H}_k^T \mathbf{R}_k^{-1} \right] \right)
        \left[ \mathbf{H}_k \mathbf{P}_{k|k-1} \mathbf{H}_k^T \mathbf{R}_k^{-1} + \mathbf{I} \right]^{-1} \\
        &= \mathbf{P}_k \mathbf{H}_k^T \mathbf{R}_k^{-1} \left[ \mathbf{H}_k \mathbf{P}_{k|k-1} \mathbf{H}_k^T \mathbf{R}_k^{-1} + \mathbf{I} \right]
        \left[ \mathbf{H}_k \mathbf{P}_{k|k-1} \mathbf{H}_k^T \mathbf{R}_k^{-1} + \mathbf{I} \right]^{-1} \\
        &= \mathbf{P}_k \mathbf{H}_k^T \mathbf{R}_k^{-1}
    \end{aligned}
\end{equation*}

and so the Kalman gain can be expressed as

\boxed{
\parbox{\textwidth}{
\begin{equation}
    \mathbf{K}_k = \mathbf{P}_k \mathbf{H}_k^T \mathbf{R}_k^{-1}
    \label{eq:kalman-gain-from-information-matrix}
\end{equation}
}
}

It is important to recognize that this form of the Kalman gain relies on the
\textit{a posteriori} covariance, $\mathbf{P}_k$. In contrast, the regular form of the
Kalman gain (\ref{eq:kalman-gain})

\begin{equation*}
    \mathbf{K}_k = \mathbf{P}_{k|k-1} \mathbf{H}_k^T \mathbf{S}_k^{-1}
\end{equation*}

relies on the \textit{a priori} covariance, $\mathbf{P}_{k|k-1}$.
Hence, the use of the \textit{a posteriori} form of the Kalman gain requires that the
\textit{a posteriori} state update must be done after the \textit{a posteriori} state
covariance update and the associated gain evaluation.

The information form of the Kalman filter update cycle for a forced input system is

I. Projection (\textit{a priori}) update:

\begin{equation*}
    \begin{aligned}
        \hat{\mathbf{x}}_{k|k-1} &= \mathbf{\Phi}_{k|k-1} \hat{\mathbf{x}}_{k-1} + \mathbf{u}_{k-1} \\
        \mathbf{P}_{k|k-1} &= \mathbf{\Phi}_{k|k-1} \, \mathbf{P}_{k-1} \, \mathbf{\Phi}_{k|k-1}^T + \mathbf{\Gamma}_{k|k-1} \mathbf{Q}_{k-1} \mathbf{\Gamma}_{k|k-1}^T
    \end{aligned}
\end{equation*}

II. Correction (\textit{a posteriori}) update:

\begin{equation*}
    \begin{aligned}
        \hat{\mathbf{z}}_k &= \mathbf{H}_k \hat{\mathbf{x}}_{k|k-1} \\
        \tilde{\mathbf{z}}_k &= \mathbf{z}_k - \hat{\mathbf{z}}_k \\
        \mathbf{P}_k^{-1} &= \mathbf{P}_{k|k-1}^{-1} + \mathbf{H}_k^T \mathbf{R}_k^{-1} \mathbf{H}_k \\
        \mathbf{K}_{k} &= \mathbf{P}_k \mathbf{H}_k^T \mathbf{R}_k^{-1} \\
        \hat{\mathbf{x}}_k &= \hat{\mathbf{x}}_{k|k-1} +\mathbf{K}_k \, \tilde{\mathbf{z}}_k
    \end{aligned}
\end{equation*}

Note that this requires performing an inversion of $\mathbf{P}_{k|k-1}$ and another
inversion of $\mathbf{P}_k^{-1}$. Also, instead of needing to form and compute
$\mathbf{S}_k^{-1}$, we compute $\mathbf{R}_k^{-1}$. While the size of $\mathbf{R}_k$
and $\mathbf{S}_k$ are both $m \times m$, typically $\mathbf{R}_k$ is static across
successive measurement updates and so $\mathbf{R}_k^{-1}$ needs to computed only once.
In addition, if $\mathbf{R}_k$ is diagonal, evaluation of $\mathbf{R}_k^{-1}$ is trivial.

In general, this form of the Kalman filter is not implemented unless the equations can
be coded such that they are computationally feasible and can numerically preserve the
structure of $\mathbf{P}_k$.

